\documentclass[UTF8]{ctexart}
\usepackage{xeCJK}
\usepackage[colorlinks, linkcolor=red]{hyperref}
\setCJKmainfont{SimSun}
\usepackage{listings}
\title{Deepin Linux 使用笔记}
\author{JackLovel}
\begin{document}
\maketitle
\newpage
\tableofcontents
\newpage
\section{安装}
\subsection{下载镜像}
到\href{https://mirrors.tuna.tsinghua.edu.cn/deepin-cd/}{清华源}下载系统的镜像,下载速度比官网下载速度快一点。

建议安装最新的版本的镜像.
\subsection{制作启动盘镜像}
% window 系统下 
\flushleft
\begin{enumerate}
\item 如果在 window 系统下, 可以使用 Rufus, 选好镜像后,分区类型选GPT,刻录模式为DD
\item 如果在 linux 系统下,可以使用 dd 命令。
\begin{lstlisting}
$ sudo fdisk -l 
$ umount /dev/sdc1
$ sudo mkfs.vfat /dev/sdc -I
$ sudo dd bs=4M if=/home/gog/下载/manjaro-kde-18.0.4-stable-x86_64.iso of=/dev/sdc status=progress
\end{lstlisting}

\end{enumerate}

\section{系统配置}
\subsection{更新源}
这里阿里源为例:
\begin{lstlisting}
$ sudo cp /etc/apt/sources.list /etc/apt/sources.list.bak  
$ sudo deepin-editor /etc/apt/sources.list
\end{lstlisting}
我们对原文件做了一个备份,而不是直接修改,这是一个好的习惯,\\
对于以后的 软件 配置时,如果你不是十分的有把握,就先进行备份,然后进行修改。\\


修改 sources.list 文件: \\
将 http://packages.deepin.com/deepin 替换成 http://mirrors.aliyun.com/deepin
\section{基本软件的配置}
\subsection{zsh}
\subsection{emacs}
\subsection{jdk}
\subsection{python}
\subsection{docker}
\section{基本命令}

\end{document}
