\section{基本软件的配置}
% zsh
\subsection{zsh}
由于系统原来 bash shell 不怎么好用,所以这里我们推荐使用 zsh \\

\begin{itemize}
\item 安装 zsh:
\begin{lstlisting}
$ sudo apt-get install -y zsh
$ sh -c \
 "$(curl -fsSL https://raw.github.com/robbyrussell/oh-my-zsh/master/tools/install.sh)"

# 将 zsh 更改为默认的 shell 
$ chsh -s /bin/zsh 
\end{lstlisting}

\item 安装 percol
\begin{lstlisting}
# 如果 pip 没有安装的话
$ sudo apt install -y python-pip
 
$ sudo pip install percol
$ deepin-editor ~/.zshrc 
\end{lstlisting}

\item 添加配置
\begin{lstlisting}
function exists { which $1 &> /dev/null }

if exists percol; then
    function percol_select_history() {
        local tac
        exists gtac && tac="gtac" || { exists tac && tac="tac" || { tac="tail -r" } }
        BUFFER=$(fc -l -n 1 | eval $tac | percol --query "$LBUFFER")
        CURSOR=$#BUFFER         # move cursor
        zle -R -c               # refresh
    }

    zle -N percol_select_history
    bindkey '^R' percol_select_history
fi
\end{lstlisting}
\end{itemize}
\newpage

% emacs 
\subsection{emacs}
下面介绍二种方式安装 emacs:
\flushleft
\begin{enumerate}
% apt 安装
\item 每一种方法,比较简单,但是 emacs 版本还是比较低的。
\begin{lstlisting}
$ sudo apt install -y emacs 

// 查看版本
$ emacs --version  
\end{lstlisting}

% 源码安装
\item 第二种方法,相对于第一种方法来说,略微复杂,比如: 依赖问题;
但是软件版本还是比较高的.

\begin{itemize}
\item 下载源码 \\ 
\begin{lstlisting}
https://www.gnu.org/software/emacs/download.html#gnu-linux
\end{lstlisting}

\item 解压
\begin{lstlisting}
$ tar xvf emacs-26.2.tar.gz  // 解压,并切换到解压后的目录
\end{lstlisting}

\item 安装依赖
\begin{lstlisting}
$ sudo apt-get install build-essential \
  texinfo libx11-dev libxpm-dev libjpeg-dev \
  libpng-dev libgif-dev libtiff-dev libgtk2.0-dev \
  libncurses-dev libxpm-dev automake autoconf 
\end{lstlisting}

\item 编译及安装
\begin{lstlisting}
$ ./configure --with-mailutils 
$ sudo make && sudo make install  
\end{lstlisting}

\item 检测
\begin{lstlisting}
$ emacs --version
\end{lstlisting}

\end{itemize}
\end{enumerate}
\newpage

% java jdk
\subsection{jdk}
这里安装的是 oracle jdk, 所以到 oracle 官网下载 jdk
\begin{itemize}
\item 下载
\begin{lstlisting}
https://www.oracle.com/technetwork/java/javase/downloads/jdk8-downloads-2133151.html
\end{lstlisting}


\item 解压
\begin{lstlisting}
$ tar xvf jdk-1.8.tar.gz
\end{lstlisting}

\item 添加配置,将下面的内容写入 ~/.zshrc 或者 ~/.bashrc
\begin{lstlisting}
export JAVA_HOME= 此处填写jdk的绝对路径
export JRE_HOME=${JAVA_HOME}/jre
export CLASSPATH=.:${JAVA_HOME}/lib:${JRE_HOME}/lib
export PATH=${JAVA_HOME}/bin:$PATH
\end{lstlisting}

\item 检测 
\begin{lstlisting}
$ source ~/.zshrc 

$ java -version 
$ javac 
\end{lstlisting}
\end{itemize}
\newpage

% python
\subsection{python}
\begin{itemize}
\item pip 安装
\begin{lstlisting}
# python2
$ sudo apt install python-pip
$ sudo pip --version

# python3
$ sudo apt install python3-pip
$ sudo pip3 --version
\end{lstlisting}

\item pypi配置
\begin{lstlisting}
$ mkdir ~/.pip
$ cd ~/.pip
$ touch pip.conf
$ deepin-editor ~/.pip/pip.conf 
\end{lstlisting}

然后添加下面的内容:
\begin{lstlisting}
[global]
index-url = http://pypi.douban.com/simple
[install]
trusted-host=pypi.douban.com
\end{lstlisting}

\item ipython
\begin{lstlisting}
# python2
$ sudo apt install -y ipython

# python3
$ sudo apt install -y ipython3 
\end{lstlisting}
\end{itemize}
\newpage

% qt creator
\subsection{qt creator}
\begin{itemize}
\item 下载
\href{http://iso.mirrors.ustc.edu.cn/qtproject/archive/qt/5.8/5.8.0/qt-opensource-linux-x64-5.8.0.run}{qt-opensource-linux-x64-5.8.0.run下载地址}

\item 安装
\begin{lstlisting}
$ chmod +x qt-opensource-linux-x64-5.8.0.run 
$ ./qt-opensource-linux-x64-5.8.0.run 
\end{lstlisting}

\item 解决中文输入
\begin{lstlisting}
$ cd /usr/lib/x86_64-linux-gnu/qt5/plugins/platforminputcontexts

# 我的 qt安装目录: ~/Qt5.8.0
$ cp libfcitxplatforminputcontextplugin.so \
  ~/Qt5.8.0/5.8/gcc_64/plugins/platforminputcontexts

$ cp libfcitxplatforminputcontextplugin.so \
  ~/Qt5.8.0/Tools/QtCreator/lib/Qt/plugins/platforminputcontexts
\end{lstlisting}
\end{itemize}
% docker
% pyenv 
