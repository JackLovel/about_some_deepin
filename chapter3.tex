\section{基本软件的配置}
\subsection{zsh}
由于系统原来 shell 不怎么好用,所以这里我们推荐使用 zsh \\

安装步骤:
\begin{lstlisting}
$ sudo apt-get install -y zsh
$ sh -c "$(curl -fsSL https://raw.github.com/robbyrussell/oh-my-zsh/master/tools/install.sh)"

# 将 zsh 更改为默认的 shell 
$ chsh -s /bin/zsh 
\end{lstlisting}

安装 percol
\begin{lstlisting}
$ sudo pip install percol
$ deepin-editor ~/.zshrc 
\end{lstlisting}

添加下面的内容:
\begin{verbatim}
function exists { which $1 &> /dev/null }

if exists percol; then
    function percol_select_history() {
        local tac
        exists gtac && tac="gtac" || { exists tac && tac="tac" || { tac="tail -r" } }
        BUFFER=$(fc -l -n 1 | eval $tac | percol --query "$LBUFFER")
        CURSOR=$#BUFFER         # move cursor
        zle -R -c               # refresh
    }

    zle -N percol_select_history
    bindkey '^R' percol_select_history
fi
\end{verbatim}

\subsection{emacs}
下面介绍二种方式安装 emacs:
\flushleft
\begin{enumerate}
% apt 安装
\item 每一种方法,比较简单,但是 emacs 版本还是比较低的。
\begin{lstlisting}
$ sudo apt install -y emacs 

// 查看版本
$ emacs --version  
\end{lstlisting}

% 源码安装
\item 第二种方法,相对于第一种方法来说,略微复杂,比如: 依赖问题; \\
但是软件版本还是比较高的, 但是对于新手来说 越简单的安装方式,就是最好的方式, 
这也是应该提倡的。

\begin{itemize}
\item 下载源码 \\ 
\begin{verbatim}
https://www.gnu.org/software/emacs/download.html#gnu-linux
\end{verbatim}

\item 解压
\begin{verbatim}  
$ tar xvf emacs-26.2.tar.gz  // 解压,并切换到解压后的目录
\end{verbatim}

\item 安装依赖
\begin{verbatim}  
$ sudo apt-get install build-essential \
  texinfo libx11-dev libxpm-dev libjpeg-dev \
  libpng-dev libgif-dev libtiff-dev libgtk2.0-dev \
  libncurses-dev libxpm-dev automake autoconf 
\end{verbatim}

\item 编译及安装
\begin{verbatim}  
$ ./configure --with-mailutils 
$ sudo make && sudo make install  
\end{verbatim}

\item 检测
\begin{verbatim}
$ emacs --version
\end{verbatim}

\end{itemize}
\end{enumerate}
\newpage

\subsection{jdk}
这里安装的是 oracle jdk

\flushleft
\begin{enumerate}
\begin{itemize}
\item 下载 
 
\end{itemize}
\end{enumerate}

\subsection{python}
\subsection{docker}