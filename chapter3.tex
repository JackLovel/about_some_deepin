\chapter{基本软件的配置}
\section{zsh}
由于系统原来 bash shell 不怎么好用,所以这里我们推荐使用 zsh \\

\begin{itemize}
\item 安装 zsh:
\begin{lstlisting}
$ sudo apt install -y zsh

# 将 zsh 更改为默认的 shell 
$ chsh -s /bin/zsh 
\end{lstlisting}

\item 安装 percol
\begin{lstlisting}
# 如果 pip 没有安装的话
$ sudo apt install -y python-pip
 
$ sudo pip install percol
$ gedit ~/.zshrc 
\end{lstlisting}

\item 添加配置
\begin{lstlisting}
function exists { which $1 &> /dev/null }

if exists percol; then
    function percol_select_history() {
        local tac
        exists gtac && tac="gtac" || { exists tac && tac="tac" || { tac="tail -r" } }
        BUFFER=$(fc -l -n 1 | eval $tac | percol --query "$LBUFFER")
        CURSOR=$#BUFFER         # move cursor
        zle -R -c               # refresh
    }

    zle -N percol_select_history
    bindkey '^R' percol_select_history
fi
\end{lstlisting}
\end{itemize}
\newpage

% emacs 
\section{emacs}
下面介绍二种方式安装 emacs:
\flushleft
\begin{enumerate}
% apt 安装
\item 每一种方法,比较简单,但是 emacs 版本还是比较低的。
\begin{lstlisting}
$ sudo apt install -y emacs 

// 查看版本
$ emacs --version  
\end{lstlisting}

% 源码安装
\item 第二种方法

相对于第一种方法来说,略微复杂,比如: 依赖问题;但是软件版本还是比较高的.

\begin{itemize}
\item 下载源码 

\begin{lstlisting}
https://www.gnu.org/software/emacs/download.html#gnu-linux
\end{lstlisting}

\item 解压
\begin{lstlisting}
$ tar xvf emacs-27.1.tar.gz  // 解压,并切换到解压后的目录
\end{lstlisting}

\item 安装依赖
\begin{lstlisting}
$ sudo apt install build-essential \
  texinfo libx11-dev libxpm-dev libjpeg-dev \
  libpng-dev libgif-dev libtiff-dev libgtk2.0-dev \
  libncurses-dev libxpm-dev automake autoconf gnutls-dev
\end{lstlisting}

\item 编译及安装
\begin{lstlisting}
$ ./configure --with-mailutils 
$ sudo make -j8 && sudo make install -j8
\end{lstlisting}

\item 检测
\begin{lstlisting}
$ emacs --version
\end{lstlisting}

\end{itemize}
\end{enumerate}
\newpage

% java jdk
\section{jdk}

\setlength\parindent{2em}这里安装的是 oracle jdk, 所以到 oracle 官网下载 jdk
\begin{itemize}
\item 下载
\begin{lstlisting}
https://www.oracle.com/technetwork/java/javase/downloads/jdk12-downloads-5295953.html
\end{lstlisting}


\item 解压
\begin{lstlisting}
$ tar xvf jdk-12.0.2_linux-x64_bin.tar.gz
\end{lstlisting}

\item 添加配置,将下面的内容写入 ~/.zshrc 或者 ~/.bashrc
\begin{lstlisting}
export JAVA_HOME= 此处填写jdk的绝对路径
export JRE_HOME=${JAVA_HOME}/jre
export CLASSPATH=.:${JAVA_HOME}/lib:${JRE_HOME}/lib
export PATH=${JAVA_HOME}/bin:$PATH
\end{lstlisting}

\item 检测 
\begin{lstlisting}
$ source ~/.zshrc 

$ java -version 
$ javac 
\end{lstlisting}
\end{itemize}
\newpage

% python
\section{python}
\begin{itemize}
\item pip 安装
\begin{lstlisting}
# python2
$ sudo apt install python-pip
$ sudo pip --version

# python3
$ sudo apt install python3-pip
$ sudo pip3 --version
\end{lstlisting}

\item pypi配置
\begin{lstlisting}
$ mkdir ~/.pip
$ cd ~/.pip
$ touch pip.conf
$ deepin-editor ~/.pip/pip.conf 
\end{lstlisting}

然后添加下面的内容:
\begin{lstlisting}
[global]
index-url = http://pypi.douban.com/simple
[install]
trusted-host=pypi.douban.com
\end{lstlisting}

\item ipython
\begin{lstlisting}
# python2
$ sudo apt install -y ipython

# python3
$ sudo apt install -y ipython3 
\end{lstlisting}

\item pyenv   

python 版本管理工具    
\begin{lstlisting}
# 下载 pyenv
$ git clone https://github.com/pyenv/pyenv.git ~/.pyenv

# 配置环境
$ echo 'export PYENV_ROOT="$HOME/.pyenv"' >> ~/.zshrc
$ echo 'export PATH="$PYENV_ROOT/bin:$PATH"' >> ~/.zshrc
$ echo 'eval "$(pyenv init -)"' >> ~/.zshrc

# 使配置生效
$ source ~/.zshrc

# 检测
$ pyenv --help

# deepin 系统中不建议 删除 原有的python版本,具体原因这里就不细说了。
# 这里我举一个 安装 python3.7.4 版本的过程:
$ pyenv install -v 3.7.4
$ pyenv global 3.7.4 # 设置系统中 python 版本
$ pyenv versions # 查看当前系统 python 的版本

\end{lstlisting}
\end{itemize}
\newpage

% qt creator
\section{qt creator}
\begin{itemize}
\item 下载 \\
\href{https://mirrors.tuna.tsinghua.edu.cn/qt/official_releases/qt/5.14/5.14.2/qt-opensource-linux-x64-5.14.2.run}{qt creator下载地址}

\item 安装
\begin{lstlisting}
$ sudo apt install -y  libgl1-mesa-dev
$ sudo chmod +x qt-opensource-linux-x64-5.14.2.run
$ sudo ./qt-opensource-linux-x64-5.8.0.run 
\end{lstlisting}

%\item 解决中文输入
%\begin{lstlisting}
%
%\end{lstlisting}
\end{itemize}
\newpage

% tex live 2020
\section{latex}
tex live 安装 \\

\begin{itemize}
\item 下载
\begin{lstlisting}
// 下载 texlive2020.iso
https://mirrors.tuna.tsinghua.edu.cn/CTAN/systems/texlive/Images/
\end{lstlisting}

\item 安装 latex:
\begin{lstlisting}
// 首先,解压 镜像
// 然后安装
$ chmod +x install-tl
$ sudo ./install-tl
\end{lstlisting}

\item 添加环境
 \begin{lstlisting}
 $ sudo gedit ~/.bashrc
 \end{lstlisting}
  
添加下面的内容
\begin{lstlisting}
export PATH=/usr/local/texlive/2020/bin/x86_64-linux:$PATH
export MANPATH=/usr/local/texlive/2020/texmf-dist/doc/man:$MANPATH
export INFOPATH=/usr/local/texlive/2020/texmf-dist/doc/info:$INFOPATH
\end{lstlisting}

\item 测试
\begin{lstlisting}
$ source ~/.bashrc
$ tex -v
\end{lstlisting}
\end{itemize}
\newpage

% mysql 8.0 
\section{mysql}

\begin{itemize}
\item 安装:
\begin{lstlisting}
$ sudo apt install mysql-server -y
\end{lstlisting}

\item 初始化数据库
\begin{lstlisting}
$ sudo mysql_secure_installation

// 参考
Securing the MySQL server deployment.

Connecting to MySQL using a blank password.

VALIDATE PASSWORD COMPONENT can be used to test passwords
and improve security. It checks the strength of password
and allows the users to set only those passwords which are
secure enough. Would you like to setup VALIDATE PASSWORD component?

Press y|Y for Yes, any other key for No: N
Please set the password for root here.

New password: 

Re-enter new password: 
By default, a MySQL installation has an anonymous user,
allowing anyone to log into MySQL without having to have
a user account created for them. This is intended only for
testing, and to make the installation go a bit smoother.
You should remove them before moving into a production
environment.

Remove anonymous users? (Press y|Y for Yes, any other key for No) : N

 ... skipping.

Normally, root should only be allowed to connect from
'localhost'. This ensures that someone cannot guess at
the root password from the network.

Disallow root login remotely? (Press y|Y for Yes, any other key for No) : Y
Success.

By default, MySQL comes with a database named 'test' that
anyone can access. This is also intended only for testing,
and should be removed before moving into a production
environment.

Remove test database and access to it? (Press y|Y for Yes, any other key for No) : N

 ... skipping.
Reloading the privilege tables will ensure that all changes
made so far will take effect immediately.

Reload privilege tables now? (Press y|Y for Yes, any other key for No) : Y
Success.

All done! 
\end{lstlisting}

\item 创建数据库
\begin{lstlisting}
$ sudo mysql -hlocalhost -uroot -p

// 更改加密方式
$ ALTER USER 'root'@'localhost' IDENTIFIED BY 数据库root字符串 PASSWORD EXPIRE NEVER;

// 更改密码
$ ALTER USER 'root'@'localhost' IDENTIFIED WITH mysql_native_password BY 数据库root字符串;

// 刷新权限
$ FLUSH PRIVILEGES;
\end{lstlisting}

\item 验证
\begin{lstlisting}
$ mysql -hlocalhost -uroot -p
\end{lstlisting}
\end{itemize}


% docker 
\section{docker}

\begin{itemize}
\item 安装:
\begin{lstlisting}
// 添加Docker官方GPG key
$ curl -fsSL https://download.docker.com/linux/ubuntu/gpg | sudo apt-key add -

// 添加仓库
$ sudo add-apt-repository \
  "deb [arch=amd64] https://download.docker.com/linux/ubuntu \
  $(lsb_release -cs) stable"

// 更新 
$ sudo apt update

// 安装 docker-ce 
$ sudo apt install docker-ce

// 检查是否安装成功 
$ sudo docker run hello-world

// 设置用户权限, 把当前用户添加 docker 用户组里
$ sudo usermod -a -G docker $USER

\end{lstlisting}

\item 添加国内源 
登录 \href{https://cr.console.aliyun.com/cn-hangzhou/instances/mirrors}{阿里源 docker 镜像服务},然后点击 镜像中心->镜像加速器
\begin{lstlisting}
// 仅供参考,我的阿里源加速器配置,有可能加速器地址不一致
$ sudo mkdir -p /etc/docker
$ sudo tee /etc/docker/daemon.json <<-'EOF'
  {
    "registry-mirrors": ["https://00o606tc.mirror.aliyuncs.com"]
  }
  EOF
$ sudo systemctl daemon-reload
$ sudo systemctl restart docker
\end{lstlisting}

\item 测试
来下载一个镜像,来测试一下, 添加的国内源是否生效
\begin{lstlisting}
// 拉取 ubuntu 镜像
$ docker pull ubuntu

// 运行以 ubuntu 为镜像,并命名为 ubuntu-test 的容器 
$ docker run -itd --name ubuntu-test ubuntu
\end{lstlisting}
\end{itemize}
