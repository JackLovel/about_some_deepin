% todo
%\section{基本命令}

\chapter{命令补充}


\section{打印系统相关的信息}
\begin{itemize}
\item 安装:
\begin{lstlisting}
# ubuntu 
$ sudo apt-get install -y screenfetch 

# opensuse 
$ sudo zypper in screenfetch
\end{lstlisting}

\item 运行:
\begin{lstlisting}
$ sudo screenfetch
\end{lstlisting}

% screenfetch  
% h: 当前位置(here)。将图形放置在正文文本中给出该图形环境的地方。如果本页所剩的页面不够,这一参数将不起作用。
% t: 顶部(top)。将图形放置在页面的顶部
% b: 底部(button)。将图形放置在页面的底部
% p: 独立一页(page)。将图形放置在一只允许有浮动对象的页面上。
\begin{figure}[htp]  
    \centering
    \includegraphics[width=0.8\textwidth]{./img/screenfetch.png} % 
    \caption{打印信息} %caption是图片的标题
    \label{screenfetch} %此处的label相当于一个图片的专属标志,目的是方便上下文的引用
\end{figure}
\end{itemize}